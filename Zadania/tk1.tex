\documentclass[10pt]{amsart}

\usepackage{amssymb, latexsym, amscd,stmaryrd}
\usepackage{amsfonts, polski}
\usepackage[utf8]{inputenc}

\theoremstyle{plain}
\newtheorem{lemma}{Lemma}
\newtheorem{theorem}[lemma]{Theorem}
\newtheorem{corollary}[lemma]{Corollary}
\newtheorem{fact}[lemma]{Fact}
\newtheorem{proposition}[lemma]{Proposition}

\theoremstyle{definition}
\newtheorem{problem}{Problem}
\newtheorem{definition}[lemma]{Definition}
\newtheorem{example}[lemma]{Example}
\newtheorem{remark}[lemma]{Remark}
\numberwithin{equation}{section}

\textwidth\paperwidth \advance\textwidth -60mm \oddsidemargin-1in
%\advance\oddsidemargin 30mm
\advance\oddsidemargin 25mm \evensidemargin-1in
\advance\evensidemargin 25mm \topmargin -1in
 %\advance\topmargin 1cm
\advance\topmargin 10mm \textheight = 700pt
\newcommand\blfootnote[1]{%
  \begingroup
  \renewcommand\thefootnote{}\footnote{#1}%
  \addtocounter{footnote}{-1}%
  \endgroup
}

\title{Teoria kategorii}
\author{Seria 1: Kategorie i funktory}
\begin{document}
\maketitle

\blfootnote{\today}
\begin{problem}
Pokazać, że $\mathsf{Rel}\cong \mathsf{Rel}^{op}$. 
\end{problem}


\begin{problem}
Niech $A$ będzie zbiorem. Pokazać, że przyporządkowania $\mathsf{Set}\to \mathsf{Set}$ zdefiniowane na obiektach i morfizmach jak poniżej są funktorami:
\begin{itemize}
\item $X\mapsto A\times X$ oraz $(f:X\to Y)\mapsto \left ((id \times f):A\times X\to A\times Y; (a,x)\mapsto (a,f(x))\right )$,
\item $X\mapsto A+ X$ oraz $$(f:X\to Y)\mapsto (id + f):A+ X\to A+ Y; x \mapsto \left \{ \begin{array}{cc}f(x)  &\text{ jeśli }x\in X, \\ x &\text{ jeśli }x\in A \end{array}\right. $$
\item $X\mapsto X^A$ oraz 
$$
f:X\to Y\quad \mapsto \quad f^A:X^A\to Y^A; \phi\mapsto f\circ \phi. 
$$
\item $X\mapsto \mathcal{P}X\stackrel{def}{=} \{ A\subseteq X \}$ oraz 
$$
(f:X\to Y)\quad \mapsto \quad \mathcal{P}(f):\mathcal{P}(X)\to \mathcal{P}(Y); A\mapsto f(A). 
$$
\end{itemize}
\end{problem}
%
%\begin{problem}
%
%\end{problem}

\begin{problem}
Pokazać, że kategoria $\mathsf{Set}$ nie jest izomorficzna z kategorią $\mathsf{Set}^{op}$. 
\end{problem}
%
%\begin{problem}
%Pokazać, że każdy izomorfizm jest mono i epi.
%\end{problem}
%
%\begin{problem}
%Pokazać, że w $\mathsf{Set}$ przekształcenie $f:X\to Y$ jest "na" wtedy i tylko wtedy, gdy jest epimorfizmem.
%\end{problem}
%
%
%\begin{problem}
%Niech $f:X\to Y$, $g:Y\to Z$ oraz $h:X\to Z$ spełniają $h= g\circ f$. Pokazać, że 
%\begin{itemize}
%\item jeśli $f,g$ są izo to $h$ też jest izo,
%\item jeśli $h$ jest mono, to $f$ jest mono,
%\item jeśli $h$ jest mono to $g$ \emph{nie} musi być mono.
%\end{itemize}
%\end{problem}
%
%
%\begin{problem}
%Pokazać, że każde dwa obiekty początkowe w danej kategorii są izomorficzne. 
%\end{problem}
%
%\begin{problem}x
%Znaleźć obiekty początkowe i końcowe (jeśli istnieją) w następujących kategoriach:
%$
%\mathsf{Rel},  \mathsf{Mon},\mathsf{CoAlg}(\Sigma\times \mathcal{I}d), \mathsf{Alg}(\Sigma\times \mathcal{I}d+1).
%$\footnote{Niech $F:\mathsf{C}\to \mathsf{C}$ będzie funktorem. $F$-algebrą nazywamy parę $(A,a:FA\to A)$ dla obiektu $A\in \mathsf{C}$. Dla dwóch $F$-algebr $(A,a:FA\to A)$, $(B,b:FB\to B)$ morfizm $h:A\to B$ nazywamy \emph{homomorfizmem} o dziedzinie $(A,a:FA\to A)$ i przeciwdziedzinie $(B,b:FB\to B)$ jeśli $h\circ a = b\circ F(h)$.  Kategorię wszystkich $F$-algebr i homomorfizmów miedzy nimi oznaczamy przez $\mathsf{Alg}(F)$. Kategorię $F$-\emph{koalgebr} $\mathsf{CoAlg}(F)$ definiujemy dualnie.}
%\end{problem}
%
%\begin{problem}
%Zdefiniować korekurencyjnie funkcję $\mathsf{merge}:\Sigma^\omega \times \Sigma^\omega \to \Sigma^\omega$, która każdej parze $(\sigma,\tau)=(\sigma_0\sigma_1\ldots,\tau_0\tau_1\ldots)$ nieskończonych ciągów nad alfabetem $\Sigma$ przyporządkowuje ciąg $$\mathsf{merge}(\sigma,\tau) = \sigma_0\tau_0\sigma_1\tau_1\ldots.$$
%\noindent \emph{Podpowiedź:} Jaki jest obiekt końcowy w $\mathsf{CoAlg}(\Sigma\times \mathcal{I}d)$?    
%\end{problem}
%
%\begin{problem}
%Niech $F:\mathsf{C}\to \mathsf{C}$ będzie funktorem. Udowodnić, że jeśli $(A,a:FA\to A)$ jest obiektem początkowym w $\mathsf{Alg}(F)$ to morfizm $a:FA\to A$ jest izomorfizmem w $\mathsf{C}$. Czy istnieje obiekt początkowy w $\mathsf{Alg}(\mathcal{P})$?
%\end{problem}
%
%
%\begin{problem}
%Pokazać, że w każdej kategorii w której mamy skończone produkty zachodzi:
%$$
%A\times (B\times C) \cong (A\times B)\times C
%$$
%\end{problem}
%
%
%\begin{problem}
%Podać definicję uogólnionego produktu dla dowolnej rodziny $\{X_i\}_{i\in I}$ obiektów z kategorii $\mathsf{C}$. 
%\end{problem}
\end{document}
