\documentclass[10pt]{amsart}

\usepackage{amssymb, latexsym, amscd,stmaryrd}
\usepackage{amsfonts, polski}
\usepackage[utf8]{inputenc}

\theoremstyle{plain}
\newtheorem{lemma}{Lemma}
\newtheorem{theorem}[lemma]{Theorem}
\newtheorem{corollary}[lemma]{Corollary}
\newtheorem{fact}[lemma]{Fact}
\newtheorem{proposition}[lemma]{Proposition}

\theoremstyle{definition}
\newtheorem{problem}{Problem}
\newtheorem{definition}[lemma]{Definition}
\newtheorem{example}[lemma]{Example}
\newtheorem{remark}[lemma]{Remark}
\numberwithin{equation}{section}

\textwidth\paperwidth \advance\textwidth -60mm \oddsidemargin-1in
%\advance\oddsidemargin 30mm
\advance\oddsidemargin 25mm \evensidemargin-1in
\advance\evensidemargin 25mm \topmargin -1in
 %\advance\topmargin 1cm
\advance\topmargin 10mm \textheight = 700pt

\newcommand\blfootnote[1]{%
  \begingroup
  \renewcommand\thefootnote{}\footnote{#1}%
  \addtocounter{footnote}{-1}%
  \endgroup
}


\title{Teoria kategorii}
\author{Seria 4: Kategorie kartezjańsko-domknięte}
\begin{document}
\maketitle

\blfootnote{\today}
\begin{problem}
Niech $\mathbb{C}$ będzie kategorią CCC. Pokazać, że $\tilde{f} = f^A\circ \eta$, gdzie dla $f:Z\times A\to B$ 
strzałka $\tilde{f}:Z\to B^A$ oznaca transpozycję, $f^A:(Z\times A)^A\to B^A$ oraz $\eta:Z\to (Z\times A)^A$ są zdefiniowane 
jak na wykładzie.
\end{problem}

\begin{problem}
Pokazać, że w dowolnej kategorii, która jest CCC zachodzi:
\begin{itemize}
\item $(A\times B)^C \cong A^C\times B^C$,
\item $(A^B)^C \cong A^{B\times C}$.
\end{itemize}
\end{problem}

\begin{problem}
Czy kategoria $\mathsf{Mon}$ jest CCC?
\end{problem}

%\begin{problem}
%Opisać kategorię $Fun(\mathsf{C},\mathsf{D})$, gdy $\mathsf{C}$ i $\mathsf{D}$ są kategoriami zadanymi przez pewne posety.
%\end{problem}

%
%\begin{problem}
%Pokazać, że następujące rodziny są transformacjami naturalnymi między odpowiednimi $\mathsf{Set}$-funktorami:
%\begin{itemize}
%\item $\{\eta_X:X\to \mathcal{P}X\}_{X\in \mathsf{Set}}$, gdzie $\eta_X(x) = \{x\}$,
%\item $\{\mu_X: \mathcal{P}\mathcal{P}X\to \mathcal{P}X\}_{X}, \mu_X(S) =\bigcup S$,
%\item $\{\mu_{X}:M\times M\times X\to M\times X\}_{X}$, gdzie $M=(M,\cdot,1)$ jest ustalonym z góry monoidem i $\mu_{X}(m,n,x) = (m\cdot n,x)$,
%\end{itemize}
%\end{problem}

\begin{problem}
Pokazać, że kategoria $\omega \mathsf{CPO}$ jest CCC, natomiast kategoria $\omega \mathsf{CPO}_\perp$ nie jest CCC.\footnote{Poset $(P,\leq)$ nazywamy $\omega CPO$ jeśli każdy przeliczalny łańcuch $x_1\leq x_2 \leq \dots$ ma supremum. Przekształcenie $f:P\to Q$, które zachowuje porządek między dwoma posetami $(P,\leq)$ i $(Q,\leq)$, które dodatkowo są $\omega CPO$ nazywamy \emph{ciągłym}, jeśli zachowuje suprema przeliczalnych łańchuchów, tj. $f(\bigvee_{i\in \mathbb{N}} x_i ) = \bigvee_{i} f(x_i)$ dla każdego $x_1\leq x_2\leq \ldots$. Posety, które spełniają własność $\omega CPO$ wraz z ciągłymi przekształceniami jako morfizmami tworzą kategorię oznaczaną przez $\omega\mathsf{CPO}$. 

Poset $(P,\leq)$, który jest $\omega CPO$ nazywamy \emph{punktowym}, jeśli istnieje w nim element najmniejszy $\perp\in P$. Punktowe $\omega CPO$ tworzą kategorię w której strzałkami są wszystkie ciągłe przekształcenia dodatkowo zachowujące element najmniejszy, tj. $h(\perp)= \perp$. Tę kategorię oznaczamy przez $\omega \mathsf{CPO}_\perp$.}  
\end{problem}


\begin{problem}
Pokazać, że kategoria wszystkich małych kategorii i funktorów $\mathsf{Cat}$ jest CCC, gdzie $\mathsf{C}^\mathsf{D} = Fun(\mathsf{C},\mathsf{D})$. 
\end{problem}
%
%\begin{problem}
%Pokazać, że $Fun(\mathsf{C},\mathsf{D})$ ma binarne produkty jeśli $\mathsf{D}$ ma binarne produkty.
%\end{problem}
%
%\begin{problem}
%Znaleźć własności, które \emph{nie} są zachowywane przez równoważność kategorii.
%\end{problem}
%
%\begin{problem}
%Kategorię nazywamy \emph{szkieletową}, jeśli dowolne dwa izomorficzne obiekty są sobie równe. Pokazać, że każda kategoria jest równoważna pewnej kategorii szkieletowej. 
%\end{problem}
\end{document}
