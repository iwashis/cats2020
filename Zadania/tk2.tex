\documentclass[10pt]{amsart}

\usepackage{amssymb, latexsym, amscd,stmaryrd}
\usepackage{amsfonts, polski}
\usepackage[utf8]{inputenc}

\theoremstyle{plain}
\newtheorem{lemma}{Lemma}
\newtheorem{theorem}[lemma]{Theorem}
\newtheorem{corollary}[lemma]{Corollary}
\newtheorem{fact}[lemma]{Fact}
\newtheorem{proposition}[lemma]{Proposition}

\theoremstyle{definition}
\newtheorem{problem}{Problem}
\newtheorem{definition}[lemma]{Definition}
\newtheorem{example}[lemma]{Example}
\newtheorem{remark}[lemma]{Remark}
\numberwithin{equation}{section}

\textwidth\paperwidth \advance\textwidth -60mm \oddsidemargin-1in
%\advance\oddsidemargin 30mm
\advance\oddsidemargin 25mm \evensidemargin-1in
\advance\evensidemargin 25mm \topmargin -1in
 %\advance\topmargin 1cm
\advance\topmargin 10mm \textheight = 700pt
\newcommand\blfootnote[1]{%
  \begingroup
  \renewcommand\thefootnote{}\footnote{#1}%
  \addtocounter{footnote}{-1}%
  \endgroup
}

\title{Teoria kategorii}
\author{Seria 2: Podstawowe konstrukcje kategoryjne}
\begin{document}
\maketitle

\blfootnote{\today}


\begin{problem}
Znaleźć obiekt początkowy i końcowy (jeśli istnieją) w kategorii 
$\mathsf{Grp}$, $\mathsf{Pos}$, $\mathsf{Par}$ oraz w kategorii wszystkich algebr typu $(1,0)$. 
\end{problem}



\begin{problem}
Niech $\mathsf{C}$ będzie kategorią oraz niech $A\in \mathsf{C}$ będzie obiektem. Definiujemy przyporządkowanie $\mathsf{C}(A,-):\mathsf{C}\to \mathsf{Set}$ w następujący sposób:
$$
X\mapsto \mathsf{C}(A,X) \text{ oraz } f:X\to Y \mapsto \mathsf{C}(A,f):\mathsf{C}(A,X)\to \mathsf{C}(A,Y);g\mapsto f\circ g. 
$$
Pokazać, że $\mathsf{C}(A,-)$ jest funktorem. Udowodnić, że zachowuje on binarne produkty\footnote{Mówimy, że funktor $F:\mathsf{C}\to\mathsf{D}$ \emph{zachowuje produkty}, jeśli $F(A\times B) \cong F(A)\times F(B)$, dla dowolnych $A,B\in \mathsf{C}$ dla których istnieje produkt $A\times B$ w $\mathsf{C}$.}. 
\end{problem}

\begin{problem}
Pokazać, że w dowolnej kategorii $\mathsf{C}$ z (binarnymi) produktami dla dowolnych trzech obiektów $A,B,C$ zachodzi
$$
(A\times B)\times C \cong A\times (B\times C).
$$
\end{problem}

\begin{problem}
Dla dowolnej rodziny $(X_i)_{i\in I}$ obiektów z kategorii $\mathsf{C}$ napisać definicję produktu
$\prod_{i\in I} X_i$ poprzez własność uniwersalności uogólniając przypadek $|I|=2$.
\end{problem}

\begin{problem}
Pokazać, że binarne koprodukty wyznaczone są jednoznacznie z dokładnością do izomorfizmu.
\end{problem}


\begin{problem}
Udowodnić, że w kategorii grup abelowych $\mathsf{Ab}$ koprodukt $A+B$ dwóch grup abelowych $A,B$ spełnia $A+B \cong A\times B$.  
\end{problem}

\begin{problem}
Pokazać, że dla dwóch zbiorów $A$, $B$ koprodukt monoidów słów $A^\ast$ oraz $B^\ast$ w kategorii $\mathsf{Mon}$ istnieje i spełnia 
$$
A^\ast + B^\ast \cong (A+B)^\ast.
$$
\end{problem}


\begin{problem}
Pokazać, że jeśli $e:E\to A$ jest equalizatorem pewnej pary strzałek, to $e$ jest mono. Sformułować i udowodnić (bez używania zasady dualności) dualne twierdzenie dla koequalizatorów. 
\end{problem}

\begin{problem}
Pokazać, że kategoria grup abelowych $\mathsf{Ab}$ ma wszystkie equalizatory. Opisać je.  
\end{problem}

\begin{problem}
Pokazać, że $\mathsf{Set}$ ma wszystkie koequalizatory. Podać ich konstrukcję.
\end{problem}
\begin{problem}
Niech $f:A\to B$ będzie pewnym przekształceniem w $\mathsf{Set}$. Opisać equalizator przekształceń $f\circ \pi_1, f\circ \pi_2:A\times A\to B$ (taki typ equalizatora będziemy nazywać \emph{jądrem} $f$). Udowodnić, że dla dowolnej relacji równoważności $R$ na $A$ jądrem  przekształcenia ilorazowego $A\to A/R; a\mapsto [a]_R$ jest $R$.
\end{problem}
\begin{problem}
Niech $R\subseteq A\times A$ będzie dowolną relacją binarną oraz niech $\left < R\right >$ będzie najmniejszą relacją równoważności na $A$ zawierającą $R$. Pokazać, że przekształcenie ilorazowe $A\to A/\left < R \right >$ jest koequalizatorem dwóch rzutowań $R\rightrightarrows A$. 
\end{problem}
\end{document}
