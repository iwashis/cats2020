\documentclass[10pt]{amsart}

\usepackage{amssymb, latexsym, amscd,stmaryrd}
\usepackage{amsfonts, polski}
\usepackage[utf8]{inputenc}

\theoremstyle{plain}
\newtheorem{lemma}{Lemma}
\newtheorem{theorem}[lemma]{Theorem}
\newtheorem{corollary}[lemma]{Corollary}
\newtheorem{fact}[lemma]{Fact}
\newtheorem{proposition}[lemma]{Proposition}

\theoremstyle{definition}
\newtheorem{problem}{Problem}
\newtheorem{definition}[lemma]{Definition}
\newtheorem{example}[lemma]{Example}
\newtheorem{remark}[lemma]{Remark}
\numberwithin{equation}{section}

\textwidth\paperwidth \advance\textwidth -60mm \oddsidemargin-1in
%\advance\oddsidemargin 30mm
\advance\oddsidemargin 25mm \evensidemargin-1in
\advance\evensidemargin 25mm \topmargin -1in
 %\advance\topmargin 1cm
\advance\topmargin 10mm \textheight = 700pt

\newcommand\blfootnote[1]{%
  \begingroup
  \renewcommand\thefootnote{}\footnote{#1}%
  \addtocounter{footnote}{-1}%
  \endgroup
}


\title{Teoria kategorii}
\author{Seria 3: (Ko)granice}
\begin{document}
\maketitle

\blfootnote{\today}

\begin{problem}
W kategorii $\mathsf{Set}$ rozważmy dowolne przekształcenie $f:A\to B$ oraz zanurzenie $i_U:w:U\subseteq B$. Opisać pullback przekształceń $f$ i $i$.
\end{problem}

\begin{problem}
Pokazać, że strzałka $m:X\to Y$ jest mono wtedy i tylko wtedy, gdy $X$ wraz z parą $id:X\to X$, $id:X\to X$ jest pullbackiem pary $m:X\to Y$, $m:X\to Y$. Wywnioskować, że funktor $\mathsf{C}(A,-)$ zachowuje monomorfizmy.
\end{problem}

\begin{problem}
Pokazać, że w dowolnej kategorii dla pullbacku  $A'\stackrel{m'}{\leftarrow} M'\rightarrow M$ morfizmów $A'\stackrel{f}{\rightarrow } A\stackrel{m}{\leftarrow }M$, jeśli $m$ jest mono, to $m'$ też jest mono. 
\end{problem}

\begin{problem}
Pokazać, że jeśli $\mathsf{C}$ jest kategorią w której istnieją skończone granice oraz $F:\mathsf{C}\to \mathsf{C}$ jest funktorem to kategoria $\mathsf{Alg}(F)$ ma wszystkie skończone granice. 
\end{problem}

\begin{problem}
Pokazać, że jeśli kategoria $\mathbb{C}$ ma skończone produkty i pullbacki, to ma ekwalizatory.
\end{problem}

\begin{problem}
Zdualizować definicję pullbacku (nazywając go pushoutem). Opisać pushouty w $\mathsf{Set}$. Pokazać konstrukcję pushoutów za pomocą koproduktów i koekwalizatorów. 
\end{problem}

\begin{problem}
Podać definicję kostożka i kogranicy nad diagramem $D:J\to \mathsf{K}$. Pokazać, że kategoria $K$ ma wszystkie skończone kogranice wtedy i tylko, gdy ma wszystkie skończone koprodukty i koekwalizatory (bez używania zasady dualności). 
\end{problem}

\begin{problem}
Pokazać, że $\mathsf{C}(-,A):\mathsf{C}^{op}\to \mathsf{Set}$ przekształca koekwalizatory w $\mathsf{C}$ na ekwalizatory i obiekt początkowy na obiekt końcowy.
\end{problem}


\end{document}
