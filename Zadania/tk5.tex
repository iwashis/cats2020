\documentclass[10pt]{amsart}

\usepackage{amssymb, latexsym, amscd,stmaryrd}
\usepackage{amsfonts, polski}
\usepackage[utf8]{inputenc}

\theoremstyle{plain}
\newtheorem{lemma}{Lemma}
\newtheorem{theorem}[lemma]{Theorem}
\newtheorem{corollary}[lemma]{Corollary}
\newtheorem{fact}[lemma]{Fact}
\newtheorem{proposition}[lemma]{Proposition}

\theoremstyle{definition}
\newtheorem{problem}{Problem}
\newtheorem{definition}[lemma]{Definition}
\newtheorem{example}[lemma]{Example}
\newtheorem{remark}[lemma]{Remark}
\numberwithin{equation}{section}

\textwidth\paperwidth \advance\textwidth -60mm \oddsidemargin-1in
%\advance\oddsidemargin 30mm
\advance\oddsidemargin 25mm \evensidemargin-1in
\advance\evensidemargin 25mm \topmargin -1in
 %\advance\topmargin 1cm
\advance\topmargin 10mm \textheight = 700pt


\title{Teoria kategorii}
\author{Seria 5: Transformacje naturalne i monady}
\begin{document}
\maketitle

\begin{problem} \label{problem:1}
  Niech $\mathbb{C}$ będzie kategorią kartezjańsko-domkniętą. Pokazać, że następujące rodziny:
  \begin{itemize}
    \item $\{\eta_X: X\to (X\times A)^A\}_X$, 
    \item $\{\varepsilon_X: X^A\times A\to X\}_X$,
    \item $\{\mu_X: \left ( (X\times A)^A \times A\right )^A \to (X\times A)^A \}$, gdzie 
        $\mu_X =  (\varepsilon_{X\times A})^A$. 
  \end{itemize}
  są transformacjami naturalnymi między odpowiednimi funktorami. 
\end{problem} 

\begin{problem} \label{problem:2}
  Pokazać, że następujące rodziny $\{\eta_X\}_X$ i $\{\mu_X\}_X$ są 
transformacjami naturalnymi między odpowiednimi $\mathsf{Set}$-funktorami:
\begin{enumerate}
\item $\{\eta_X:X\to \mathcal{P}X\}_{X\in \mathsf{Set}}$, gdzie $\eta_X(x) = \{x\}$ oraz 
  $\{\mu_X: \mathcal{P}\mathcal{P}X\to \mathcal{P}X\}_{X\in \mathsf{Set}}, \mu_X(S) =\bigcup S$,
\item $\{\eta_X:X\to X^\ast\}_{X\in \mathsf{Set}}$, gdzie $\eta_X(x) = x$ 
  oraz $\{\mu_X: (X^\ast)^\ast \to X^\ast\}_{X\in \mathsf{Set}}$, 
  gdzie $\mu_X( s_1,s_2,\ldots, s_n ) = s_1s_2\ldots s_n$.
\item Niech  $M=(M,\cdot,1)$ będzie ustalonym z góry monoidem, a ponadto
  $\{\eta_X: X\to M\times X\}_{X\in \mathsf{Set}}$, gdzie $\eta_X(x) = (1,x)$ oraz  
  $\{\mu_{X}:M\times M\times X\to M\times X\}_{X\in \mathsf{Set}}$, 
  gdzie $\mu_{X}(m,n,x) = (m\cdot n,x)$.
\end{enumerate}
\end{problem}


\begin{problem} 
Pokazać, że następujące trójki są monadami:
\begin{enumerate}
  \item 
    $((\mathcal{I}d\times A)^A:\mathbb{C}\to \mathbb{C}, 
    \mu: \left ((\mathcal{I}d\times A)^A \times A\right)^A\implies (\mathcal{I}d\times A)^A, 
    \eta: \mathcal{I}d\implies (\mathcal{I}d\times A)^A)
    $, 
    gdzie $\mathbb{C}$ jest kategorią kartezjańsko domkniętą oraz $\mu$ i $\eta$ są jak w Zadaniu \ref{problem:1}.  
  \item $(T, \mu, \eta)$, gdzie $T\in \{\mathcal{P}, (-)^\ast, M\times \mathcal{I}d\}$, oraz 
      $\mu$ i $\eta$ są jak w Zadaniu \ref{problem:2} dla odpowiednich funktorów. 
   
\end{enumerate}
\end{problem}


\begin{problem}
  Niech $(T,\mu,\eta)$ będzie monadą na kategorii $\mathbb{C}$. Dla $f:A\to TB, g:B\to TC$ zdefiniujmy:
  $$ g \cdot f :A\to TC; g\cdot f = \mu_C \circ Tg \circ f .$$ 
  Pokazać, że dla $f,g$ oraz $h:C\to TD$ zachodzi:
  \begin{align*}
    & (h\cdot g)\cdot f = h\cdot (g\cdot f) \text{ oraz } f\cdot \eta_A = \eta_B\cdot f = f.  
  \end{align*}
  
\end{problem}

\begin{problem}
 Trójkę $(T,(-)^\ast, \eta)$ nazywamy \emph{trójką Kleisliego}, jeśli
 \begin{itemize}
   \item $T:\mathbb{C}\to \mathbb{C}$ jest funktorem,
   \item $\eta= \{\eta_X:X\to TX\}_{X\in \mathbb{C}}$ jest rodziną strzałek,
   \item $(-)^\ast$ przyporządkowuje dowolnej strzałce $f:X\to TY$ strzałkę $f^\ast :TX\to TY$, 
  \end{itemize}
 dodatkowo spełniającymi następujące równania:
 \[ 
   \eta_X^\ast = id_{TX}, \text{ oraz } f^\ast \circ \eta_X = f \text{ oraz } 
    (g^\ast \circ f)^\ast = g^\ast \circ f^\ast.
 \]
 Pokazać, że jeśli $(T, (-)^\ast, \eta)$ jest trójką Kleisliego, to $(T,\mu,\eta)$ jest monadą dla 
 $\mu = \{\mu_X:T^2X\to TX\}$ dla $\mu_X = (id_{TX})^\ast$. Ponadto, jeśli $(T,(-)^\ast, \eta)$ jest monadą, to $(T,(-)^\ast, \eta)$ jest trójką Kleisliego, gdzie $f^\ast = \mu_Y \circ Tf$ dla $f:X\to TY$. 
\end{problem} 

\end{document}
